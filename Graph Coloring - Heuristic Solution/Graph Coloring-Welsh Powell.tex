\documentclass[a4paper,12pt]{article}
\usepackage{amsmath}
\usepackage{xcolor}
\usepackage{lipsum}
\newcommand*{\mybox}[2]{\colorbox{#1!30}{\parbox{.98\linewidth}{#2}}}
%graphics package
\usepackage{graphicx} %to import images
\usepackage{float}%to control of the float positions

\usepackage[english]{babel}
\usepackage[utf8x]{inputenc}
\usepackage[colorinlistoftodos]{todonotes}

%programming code package
\usepackage{listings}
\usepackage{color}
\definecolor{dkgreen}{rgb}{0,0.6,0}
\definecolor{gray}{rgb}{0.5,0.5,0.5}
\definecolor{mauve}{rgb}{0.58,0,0.82}
\definecolor{lbcolor}{rgb}{0.9,0.9,0.9}

\begin{document}
\begin{titlepage}

\newcommand{\HRule}{\rule{\linewidth}{0.5mm}} % Defines a new command for the horizontal lines, change thickness here

\center % Center everything on the page
 
%----------------------------------------------------------------------------------------
%   HEADING SECTIONS
%----------------------------------------------------------------------------------------

\textsc{\LARGE International Islamic University Chittagong}\\[1.5cm] % Name of your university/college
%\textsc{\Large Major Heading}\\[0.5cm] % Major heading such as course name
%\textsc{\large Minor Heading}\\[0.5cm] % Minor heading such as course title
%----------------------------------------------------------------------------------------
%   TITLE SECTION
%----------------------------------------------------------------------------------------

\HRule \\[0.4cm]
{ \huge \bfseries Graph Coloring: A Heuristic Solution}\\[0.4cm] % Title of your document
\HRule \\[1.5cm]
 
%----------------------------------------------------------------------------------------
%   AUTHOR SECTION
%----------------------------------------------------------------------------------------

\begin{minipage}{0.4\textwidth}
\begin{flushleft} \large
\emph{Submitted By:}\\
Md. Kalim Amzad Chy\\ % Your name
ID: C151113\\
7CM
\end{flushleft}
\end{minipage}
~
\begin{minipage}{0.4\textwidth}
\begin{flushright} \large
\emph{Sbmitted To:} \\
Dr. Kazi Ashrafuzzaman % Supervisor's Name
Associate Professor, 
Dept. of CSE., CU
\end{flushright}
\end{minipage}\\[2cm]

% If you don't want a supervisor, uncomment the two lines below and remove the section above
%\Large \emph{Author:}\\
%John \textsc{Smith}\\[3cm] % Your name

%----------------------------------------------------------------------------------------
%   DATE SECTION
%----------------------------------------------------------------------------------------

%{\large \today}\\[2cm] % Date, change the \today to a set date if you want to be precise

%----------------------------------------------------------------------------------------
%   LOGO SECTION
%----------------------------------------------------------------------------------------
\includegraphics[height=1.5in]{C:/Users/Parents/Desktop/Latex/logo.png}
%\includegraphics{logo.png}\\[1cm] % Include a department/university logo - this will require the graphicx package
 
%----------------------------------------------------------------------------------------

\vfill % Fill the rest of the page with whitespace

\end{titlepage}

\newpage
\section{Introduction}
In graph theory, graph coloring is a special case of graph labeling; it is an assignment of labels traditionally called "colors" to elements of a graph subject to certain constraints. In its simplest form, it is a way of coloring the vertices of a graph such that no two adjacent vertices share the same color; this is called a vertex coloring. Similarly, an edge coloring assigns a color to each edge so that no two adjacent edges share the same color, and a face coloring of a planar graph assigns a color to each face or region so that no two faces that share a boundary have the same color.

The greedy algorithm considers the vertices in a specific order $ v_{1},...., v_{n}$ and assigns to $v_{i}$ the smallest available color not used by $v_{i}$’s neighbours among $v_{1},...., v_{i-1} $ adding a fresh color if needed. The quality of the resulting coloring depends on the chosen ordering. There exists an ordering that leads to a greedy coloring with the optimal number of $X(G)$ colors. On the other hand, greedy colorings can be arbitrarily bad; for example, the crown graph on $n$ vertices can be 2-colored, but has an ordering that leads to a greedy coloring with $n/2$ colors. If the vertices are ordered according to their degrees, the resulting greedy coloring uses at most $max_i\ min\{d(x_i)+1,i\}$ colors, at most one more than the graph’s maximum degree.
\newpage
\section{Welsh-Powell Algorithm}
In 1967 Welsh and Powell introduced in an upper bound to the chromatic number of a graph . It provides a greedy algorithm that runs on a static graph.

Welsh Powell is used to implement graph labeling; it is an assignment of labels traditionally called "colors" to elements of a graph subject to certain constraints.

The vertices are ordered according to their degrees, the resulting greedy coloring uses at most $max_i\ min\{d(x_i)+1,i\}$ colors, at most one more than the graph’s maximum degree. This heuristic is called the Welsh–Powell algorithm.
\subsection{Proof and Coloring Complexity}
The degree of a vertex $A_i$ of the graph $G$ is the number of edges having $A_i$ as an endpoint, and we will denote  it by $d_i$. Without loss of generality we assume that 
\begin{equation}
d_1 \geq d_2 \geq ...\geq d_n
\end{equation}
It is easy to show that if $k(G)$ denotes the chromatic number of $G$ then 
\begin{equation}
k(G) \leq d_1 + 1
\end{equation}
and provided G contains no $d_1$-simplices, then from (2) we know 
\begin{equation}
k(G) \leq d_1
\end{equation}

G may  always be coloured in at most $\alpha(G)$ colours where 
\begin{equation}
k(G) \leq \alpha(G) \leq max_i\ min\{d(x_i)+1,i\}
\end{equation}
\subsubsection{Theorem 1}
If $G$ is a $k-$critical graph, then
\begin{equation*}
\delta(G) \geq k-1
\end{equation*}
\textit{Proof:}
Let $ v $ be a vertex of $ G $ so that $d(v)<k-1$. Since $G$ is $k$- critical, the subgraph $G-v$
has a $(k-1)$ -colouring. As $ v $ has at most $k-2$ neighbours, these neighbours use at
most $k-2$ colours in this $(k-1)$ colouring of $G-v$ . Now, colour $ v $ with the unused
colour and this gives a $(k-1)$ colouring of $ G $. This contradicts the given assumption
that $\chi(G) = k$. Hence every vertex $ v $ has degree at least $k-1$

\subsubsection{Theorem 2}
Let $G$ be a graph and $k = max(\delta(G\prime):G\prime$ is a subgraph of $G$. Then $\chi(G) = k-1$\\

\textit{Proof:}
Let $H$ be a $k$-minimal subgraph of $G$. Then $H$ is a subgraph of $G$ and therefore $\delta(H) \leq k$. Using Theorem 1, we have,$\delta(H)\geq\chi(H)-1 = \chi(G)-1$. Thus, $\chi(G) \leq \delta(H)+1 = k+1$.
\subsubsection{Theorem 3}
Let $G$ be graph with degree sequence such that $d_1\geq d_2\geq ... \geq d_n.$ Then, $\chi(G)\leq max \{min \{ d_i+1,i\}\}$\\

\textit{Proof:}
Let $G$ be $k$-chromatic. Then, by Theorem 2, $G$ has at least $k$ vertices of degree at least $k-1$. Therefore, $d_k\geq k-1$ and $max \{min \{ d_i+1,i\}\} \geq min\{k,d_k+1=k=\chi(G)\}$.
\subsubsection{Upper bound for chromatic number}
For any graph $G$, $\chi(G) \leq \Delta(G)+1$.\\

\textit{Proof:}
Let $G$ be any graph with $n$ vertices. To prove the result, we induct on $n$. For $n=1, G=K_1$ and $\chi(G) = 1$  and $\Delta(G) = 0$. Therefore the result is true for $n = 1$.

Assume that the result is true for all graphs with $n-1$ vertices and therefore by induction hypothesis, $\chi(G) \leq \Delta(G-v)+1$. This shows that $G-v$ can be coloured by using $\Delta(G-v) +1$ colours. Since $\Delta(G)$ is the maximum degree of a vertex in $G$, vertex $V$ has at most $\Delta(G)$ neighbours in $G$. Thus these neighbours use up at most $\Delta(G)$ colours in the colouring of $G-v$. If $\Delta(G) = \Delta(G-v),$ then there is at least one colour not used by $v'$s neighbours and that can be used to colour $v$ giving a $\Delta(G)+1$ colouring for $G$.

In case $\Delta(G)=\Delta (G-v),$ then $\Delta(G-v) < \Delta(G)$. Therefore, using a new colour for $v$,
we have a $\Delta(G-v)+2$ colouring of $G$ and clearly, $\Delta(G-v) +2 \leq \Delta(G) +1$. Hence in both cases, it follows that $\chi(G) \leq \Delta(G)+1$.
\newpage
\subsection{Algorithm}

\mybox{gray}{
\begin{enumerate}
\item Find the degree of each vertex . 
\item List the vertices in order of descending valence i.e.valence degree $(v(i)) \geq degree (v(i+1)).$
\item Colour the first vertex in the list. 
\item Go down the sorted list and color every vertex not connected to the colored  vertices above the same color then cross  out all colored vertices in the list.
\item Repeat the process on the uncolored vertices with a new color-always working in descending order of degree until all in descending order of degree until all vertices are colored.
\end{enumerate}
}
\par
\subsection{Example}
\begin{figure}[H]
 	\centering
 	\includegraphics[height=2.5in]{C:/Users/Parents/Desktop/Latex/given_graph.png}
 	\caption[optional caption]{Given Graph}
 	\label{fig:given_graph}
 \end{figure}
 
 \begin{figure}[H]
 	\centering
 	\includegraphics[height=2.5in]{C:/Users/Parents/Desktop/Latex/solved_graph.png}
 	\caption[optional caption]{Colored Graph}
 	\label{fig:colored_graph}
 \end{figure}
 
 \subsection{Time Complexity}
 We iterate through the vertex list. For every vertex we iterate through the adjacent vretices. So, if no of the vertex is $V$, time complexity of welsh-powell algorithm is $O(V^2)$
\end{document}